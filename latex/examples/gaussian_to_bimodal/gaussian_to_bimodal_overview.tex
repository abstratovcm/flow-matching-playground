\documentclass[12pt,a4paper]{article}

% Include the common preamble
\usepackage[utf8]{inputenc}
\usepackage[T1]{fontenc}
\usepackage{amsmath, amsthm, amssymb, amsfonts}
\usepackage{hyperref}
\usepackage{listings}
\usepackage{xcolor}

% Set up listings for Python code (if needed)
\definecolor{codegray}{rgb}{0.5,0.5,0.5}
\definecolor{codepurple}{rgb}{0.58,0,0.82}
\definecolor{backcolour}{rgb}{0.95,0.95,0.92}
\lstdefinestyle{pythonstyle}{
    backgroundcolor=\color{backcolour},   
    commentstyle=\color{codegray},
    keywordstyle=\color{blue},
    numberstyle=\tiny\color{codegray},
    stringstyle=\color{codepurple},
    basicstyle=\ttfamily\footnotesize,
    breaklines=true,
    numbers=left,
    numbersep=5pt,
    showstringspaces=false,
    tabsize=2
}
\lstset{style=pythonstyle}

% Theorem/Definition styles
\theoremstyle{definition}
\newtheorem{definition}{Definition}[section]

\theoremstyle{plain}
\newtheorem{theorem}{Theorem}[section]


\title{Gaussian to Bimodal Example: A Didactic Overview}
\author{Your Name}
\date{\today}

\begin{document}

\maketitle

\tableofcontents
\newpage

\section{Introduction}

This document explains, in a detailed and didactic manner, how the example code transforms a base Gaussian distribution into a bimodal distribution using flow matching in PyTorch. The example focuses on learning a time-dependent velocity field that pushes samples from a simple 2D Gaussian (base distribution) toward a bimodal mixture (target distribution).

\section{Overview of the Example Code}

The Python project is organized into several modules:
\begin{itemize}
    \item \textbf{Model Definition (\texttt{src/model.py}):}  
    Defines a neural network that approximates the velocity field \( v(x,t) \). The network takes as input a 2D point \( x \) and a time scalar \( t \) and outputs a 2D velocity vector.
    
    \item \textbf{Dataset (\texttt{src/dataset.py}):}  
    Generates sample pairs \((x_0,x_1)\) on the fly, where:
    \begin{itemize}
        \item \( x_0 \) is sampled from a standard 2D Gaussian.
        \item \( x_1 \) is sampled from a bimodal distribution (a mixture of two Gaussians).
    \end{itemize}
    
    \item \textbf{Trainer (\texttt{src/trainer.py}):}  
    Implements the training loop. For each sample pair, a random time \( t \) is chosen. The interpolated point is computed as:
    \[
    x_t = (1-t)x_0 + t\,x_1,
    \]
    and the target velocity is:
    \[
    v_{\text{target}} = x_1 - x_0.
    \]
    The model is trained so that its prediction \( v(x_t,t) \) approximates \( v_{\text{target}} \) using mean squared error (MSE).
    
    \item \textbf{Visualization:}  
    The code includes functions to plot:
    \begin{enumerate}
        \item A scatter plot of a few base (\( x_0 \)) and target (\( x_1 \)) samples with connecting lines.
        \item The learned flow field (vector field) at a fixed time (e.g., \( t=0.5 \)) along with simulated trajectories computed via Euler integration.
    \end{enumerate}
\end{itemize}

\section{Detailed Explanation}

\subsection{Data Generation and Interpolation}
For each training sample:
\begin{itemize}
    \item A base sample \( x_0 \) is drawn from a 2D standard Gaussian.
    \item A target sample \( x_1 \) is drawn from a bimodal distribution.
    \item A random time \( t \in [0,1] \) is selected.
    \item The interpolated point is computed as:
    \[
    x_t = (1-t)x_0 + t\,x_1.
    \]
    \item The target velocity is defined as:
    \[
    v_{\text{target}} = x_1 - x_0.
    \]
\end{itemize}
Because the interpolation is linear, the velocity remains constant along the path from \( x_0 \) to \( x_1 \).

\subsection{Model Training}
The neural network is trained to learn the mapping:
\[
(x_t, t) \mapsto v(x_t,t)
\]
so that the predicted velocity matches the target velocity \( x_1 - x_0 \). The loss function used is the mean squared error (MSE) between the predicted and target velocities.

\subsection{Visualization of Results}
Two main visualizations are produced:
\begin{enumerate}
    \item \textbf{Sample Pairs Plot:}  
    A scatter plot showing:
    \begin{itemize}
        \item Green points for base samples \( x_0 \).
        \item Red points for target samples \( x_1 \).
    \end{itemize}
    Gray lines connect each \( x_0 \) to its corresponding \( x_1 \), illustrating the desired transformation.
    
    \item \textbf{Flow Field and Trajectories:}  
    A vector field is plotted at a fixed time (e.g., \( t=0.5 \)) using Matplotlib’s \texttt{quiver} function. Overlaid on this vector field are sample trajectories computed via Euler integration, which demonstrate how points move under the learned flow.
\end{enumerate}

\section{Conclusion}
This example demonstrates how flow matching can be used to learn a transformation from a Gaussian distribution to a bimodal distribution. The detailed explanation provided here covers data generation, model training, and visualization, giving a complete picture of the underlying process.

\end{document}
